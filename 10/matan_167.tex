\documentclass[a4paper, 16pt]{article}
\usepackage{amsmath}
\usepackage[T2A]{fontenc}
\usepackage[english, russian]{babel}
\DeclareMathSizes{10}{10}{10}{10}
\begin{document}
\thispagestyle{empty}
\begin{Large}
\noindent$\lim_{n \to \infty} f(x_{n}) = 0$ и $\lim_{n \to \infty} f(x^{'}) = 1$, а это означает, что предела
функции(5.8) при $x \to 0$ не существует.\par
\quad\textbf{3.} Пусть
\[f(x)=\frac{x^2 + x + 1}{x^2 - 2}\]
Найдем предел этой функции при $x \to \infty$. Её областью определения является множество $X = \textbf{R}\backslash\{\sqrt{2}, - \sqrt{2}\}$. Взяв какую-либо последовательность $x_{n} \in X, n=1, 2, ... ,$ $\quad\lim_{n \to \infty} f(x_{n}) = 0$,\par
\noindentбудем иметь\par
\begin{center}
    $lim_{n \to \infty} f(x_{n}) = \frac{x_{n}^2 + x_{n} + 1}{x_{n}^2 - 2} = lim_{n \to \infty} \frac{1 + \frac{1}{x_{n}} + \frac{1}{x_{n}^2}
    }{1 - \frac{2}{x_{n}^2}}=\frac{1 + lim_{n \to \infty} \frac{1}{x_{n}} + lim_{n \to \infty} \frac{1}{x_{n}^2}{1 - 2}}{lim_{n \to \infty} \frac{1}{x_{n}^2}}=1$.\par
\end{center}
Отсюда следует, что $\lim_{n \to \infty}{\frac{x^2 + x + 1}{x^2 - 2}}=1.$\par
\begin{large}
    \noindent\textit{Упражнение 5.}
\end{large}
\begin{small}
    Доказать, что предел $\lim_{n \to \infty}{\frac{x}{\sqrt{x^2 + 1}}}$ не существует, а
    пределы $\lim_{n \to +\infty}{\frac{x}{\sqrt{x^2 + 1}}}$ и $\lim_{n \to -\infty}{\frac{x}{\sqrt{x^2 + 1}}}$ существуют, и найти их.\par
\end{small}
При рассмотрении пределов функции часто приходится иметь дело спределами сужений функций на том или ином множестве,
т. е. с пределами функций: получающихся из данных функций, рассмотрением их не на всем множестве, на котором они заданы,
а на каком-то, содержащемся в нем.\\
\textbf{Определение 3.}
\textit{Пусть $f:X \to \textbf{R}$. Предел в точке $x_0$ сужения $f_E:E \to \textbf{R}, E \subset X$ функции f на множество E называется
пределом функции f по множеству E в этой точке и обозначается через}\par
\begin{center}
    $\lim_{x \to x_0 \atop x \in E} f(x)$.
\end{center}
Таким образом,
\begin{flushright}
    $\lim_{x \to x_0 \atop x \in E} f(x)\overset{def}{=}\lim_{x \to x_0} f_E(x),$ \phantom{\hspace{55pt}}(5.9)\par
\end{flushright}
\begin{center}
    \rule{50pt}{0.1pt}\par
    \textit{167}
\end{center}
\end{Large}
\end{document}




        
